\documentclass[11pt, letterpaper]{article}
\usepackage{amsmath,amsfonts}
\usepackage{mathtools}
\usepackage{fullpage}
\usepackage{tikz,graphicx}
\usetikzlibrary{automata, positioning, arrows, shapes}
\usepackage[margin=1in]{geometry}
\usepackage[utf8]{inputenc}
\usepackage[english]{babel}
\usepackage{amsthm}

% Use listings to insert the cpp code
\usepackage{listings}
\usepackage{color}
\usepackage{xcolor}
\definecolor{dkgreen}{rgb}{0,0.6,0}
\definecolor{gray}{rgb}{0.5,0.5,0.5}
\definecolor{mauve}{rgb}{0.58,0,0.82}
\lstset{frame=tb,
     language=C++,
     aboveskip=3mm,
     belowskip=3mm,
     showstringspaces=false,
     columns=flexible,
     basicstyle = \ttfamily\small,
     numbers=left,
     numberstyle=\tiny\color{gray},
     keywordstyle=\color{blue},
     commentstyle=\color{dkgreen},
     stringstyle=\color{mauve},
     breaklines=true,
     breakatwhitespace=true,
     tabsize=3
}


\begin{document}

\title{Luogu P1226 \\ Solution}
\date{}

\maketitle

% Put the problem here
\textbf{Problem. }Given three integers $b, p, k$, query $b^{p}\bmod k$, where $0 \leq b, p < 2^{32}$ and $1 \leq k \leq 2^{32}$.\\\\
% Put the solution here
\textbf{Solution. }Since there is only one case, just use quick modular power to obtain the answer in $\mathcal{O}(\log k)$. But here we are going to talk about another technique. That is, suppose there are $5 \times 10^{7}$ cases to test with different values of $p$ and we also suppose $k$ is a constant prime, while other restrictions are not changed. \\In that case, we are going to use modular power to obtain $a^{n}\bmod{m}$ with $\Theta(\lceil \sqrt{m} \rceil)$
pre-process and $\mathcal{O}(1)$ query for different $n$.\\\\
\emph{Precondition: }
\begin{itemize}
\item $a$ and $m$ are constants.
\item $m$ is a prime.
\end{itemize}
\emph{Descriptions: }We let $k = \lceil \sqrt{m} \rceil$, then we can pre-calculate all values below in $\Theta(k)$
time:
\begin{align*}
p_{0}   &=  1 \bmod m \\
p_{1}   &=  a \bmod m\\
p_{2}   &= a^{2} \bmod m\\
& \vdots \\
p_{k-1} &= a^{k-1} \bmod m\\
p_{k}   &= a^{k} \bmod m
\end{align*}
And
\begin{align*}
q_{0}   &=  1 \bmod m\\
q_{1}   &= a^{k} \bmod m\\
q_{2}   &= a^{2k} \bmod m\\
& \vdots \\
q_{k-1} &= a^{(k-1)k} \bmod m\\
q_{k}   &= a^{k \cdot k} \bmod m
\end{align*}
Where we can obtain $p_{0}, p_{1}, \cdots , p_{k-1}, p_{k}$ and $q_{0}, q_{1}, \cdots , q_{k-1}, q_{k}$ by using recurrence relations:

\[
p_{i} =
\begin{cases}
    ap_{i-1} \bmod m & \text{if $i > 0$} \\
    1 \bmod m & \text{if $i = 0$}
\end{cases}
\]

\[
q_{i} =
\begin{cases}
    a^{k}q_{i-1} \bmod m & \text{if $i > 0$} \\
    1 \bmod m & \text{if $i = 0$}
\end{cases}
\]

Since $m$ is a prime, $a^{n}\bmod{m} = a^{n \bmod (m-1)}\bmod{m}$ due to Fermat's little theorem. Thus for each $n$ given, we suppose $n^{\prime}$ is the result of  $n \bmod (m-1)$, and we can also suppose that $n^{\prime}=xk+y$, then $x=n^{\prime} / k$ and $y=n^{\prime} \bmod k$, And we have


\begin{equation}\label{eq1}
\begin{split}
a^{n}\bmod{m}      & = a^{n \bmod (m-1)}\bmod{m} \\
                  & = a^{n^{\prime}}\bmod{m} \\
                  & = a^{xk+y}\bmod{m} \\
                  & = a^{xk}a^{y}\bmod{m} \\
                  & = (a^{xk} \bmod m)(a^{y} \bmod m) \bmod m \\
                  & = q_{x}p_{y}\bmod{m} \\
\end{split}
\end{equation}
Since we can do $q_{x}p_{y}\bmod{m}$ in $\mathcal{O}(1)$, we can solve the problem with the time complexity of $\mathcal{O}(\sqrt{m} + n)$.
Notice that if $m$ is not a prime, we can also use the conclusion below to process $n$:

\[
a^n\equiv
\begin{cases}
a^{n\bmod\varphi(m)},\,&\gcd(a,\,m)=1\\
a^n,&\gcd(a,\,m)\ne1,\,n<\varphi(m)\\
a^{n\bmod\varphi(m)+\varphi(m)},&\gcd(a,\,m)\ne1,\,n\ge\varphi(m)
\end{cases}
\pmod m
\]
Thus we can use $n \bmod \varphi(m)$ instead of $n$.
Now we wrap the algorithm with a tester in \emph{Cpp} as below:

\begin{lstlisting}[language=C++]
#include <bits/extc++.h>

using namespace std;
using namespace __gnu_pbds;
using ll = long long;

constexpr ll MOD = 1000000007;

namespace QuickModularPowerWithPreprocess {
    // all variables
    ll k;
    vector<ll> p;
    vector<ll> q;

    // O(1) modulo mul to calculate a^b mod m
    inline static
    ll modmul(ll a, ll b, ll m) {
        a = (a % m + m) % m;
        b = (b % m + m) % m;
        return ((a * b -
                 static_cast<ll>(static_cast<long double>(a) / m * b) * m) %
                m + m) % m;
    }

    // Preprocess
    inline static
    void init(ll a, ll m) {
        k = ceil(sqrt(static_cast<double>(m)));
        vector<ll>().swap(p);
        vector<ll>().swap(q);

        p.resize(k + 1);
        p.at(0) = 1 % m;
        for (int i = 1; i <= k; ++i) {
            p.at(i) = modmul(a, p.at(i - 1), m);
        }
        q.resize(k + 1);
        q.at(0) = 1 % m;
        for (int i = 1; i <= k; ++i) {
            // q[i] = a^k * q_{i-1} mod m = p.at(k) * q_{i-1} mod m
            q.at(i) = modmul(p.at(k), q.at(i - 1), m);
        }
    }

    // Query in O(1)
    inline static
    ll query(ll n) {
        ll n_prime = n % (MOD - 1);
        ll x = n_prime / k;
        ll y = n_prime % k;
        return modmul(q.at(x), p.at(y), MOD);
    }
}

namespace Tester {
    inline static
    ll modpow(ll a, ll n, ll M = MOD) {
        if (M == 1) return 0;
        ll r;
        for (r = 1, a %= M; n; a = (a * a) % M, n >>= 1)
            if (n % 2)
                r = (r * a) % M;
        return r;
    }

    auto get_random_n = []() {
        static ll lower_bound = 1;
        static ll upper_bound = 1000000000000000000;
        static random_device dev;
        static mt19937 random_generator(dev());
        static uniform_int_distribution dist(lower_bound, upper_bound);
        return dist(random_generator);
    };
}

int main() {

    int t = 10000000;
    ll a = 1147483648;
    ll b;
    // init
    QuickModularPowerWithPreprocess::init(a, MOD);
    while (t--) {
        b = Tester::get_random_n();
        assert(QuickModularPowerWithPreprocess::query(b) ==
               Tester::modpow(a, b, MOD));
    }
    return 0;
}
\end{lstlisting}

\end{document}
